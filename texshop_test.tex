\documentclass[12pt,a4paper]{article}

% ========================================
% PACKAGE IMPORTS - Testing Various TeXShop Packages
% ========================================

% Essential packages
\usepackage[utf8]{inputenc}
\usepackage[T1]{fontenc}
\usepackage[english]{babel}

% Math packages
\usepackage{amsmath}
\usepackage{amssymb}
\usepackage{amsthm}
\usepackage{mathtools}

% Graphics and figures
\usepackage{graphicx}
\usepackage{float}
\usepackage{subcaption}
\usepackage{tikz}
\usetikzlibrary{shapes,arrows,positioning}

% Tables and formatting
\usepackage{booktabs}
\usepackage{array}
\usepackage{longtable}
\usepackage{multirow}

% Colors and styling
\usepackage{xcolor}
\usepackage{fancyhdr}
\usepackage{geometry}

% Bibliography and references
\usepackage{hyperref}
\usepackage{url}
\usepackage{cite}

% Code listings
\usepackage{listings}
\usepackage{verbatim}

% Additional useful packages
\usepackage{enumerate}
\usepackage{enumitem}
\usepackage{lipsum}

% ========================================
% DOCUMENT SETUP
% ========================================

\geometry{margin=1in}
\pagestyle{fancy}
\fancyhf{}
\fancyhead[L]{TeXShop Package Test}
\fancyhead[R]{\thepage}

\hypersetup{
    colorlinks=true,
    linkcolor=blue,
    filecolor=magenta,      
    urlcolor=cyan,
    citecolor=red
}

% Define custom colors
\definecolor{codegreen}{rgb}{0,0.6,0}
\definecolor{codegray}{rgb}{0.5,0.5,0.5}
\definecolor{codepurple}{rgb}{0.58,0,0.82}
\definecolor{backcolour}{rgb}{0.95,0.95,0.92}

% Configure code listings
\lstdefinestyle{mystyle}{
    backgroundcolor=\color{backcolour},   
    commentstyle=\color{codegreen},
    keywordstyle=\color{magenta},
    numberstyle=\tiny\color{codegray},
    stringstyle=\color{codepurple},
    basicstyle=\ttfamily\footnotesize,
    breakatwhitespace=false,         
    breaklines=true,                 
    captionpos=b,                    
    keepspaces=true,                 
    numbers=left,                    
    numbersep=5pt,                  
    showspaces=false,                
    showstringspaces=false,
    showtabs=false,                  
    tabsize=2
}
\lstset{style=mystyle}

% Math theorem environments
\newtheorem{theorem}{Theorem}[section]
\newtheorem{lemma}[theorem]{Lemma}
\newtheorem{proposition}[theorem]{Proposition}
\newtheorem{corollary}[theorem]{Corollary}
\newtheorem{definition}[theorem]{Definition}

% ========================================
% DOCUMENT CONTENT
% ========================================

\title{\textbf{TeXShop Package Test Document}}
\author{Generated for Package Testing}
\date{\today}

\begin{document}

\maketitle

\tableofcontents
\newpage

% ========================================
\section{Introduction}
% ========================================

This document serves as a comprehensive test for various \LaTeX\ packages commonly used with TeXShop. It demonstrates mathematical typesetting, graphics, tables, code listings, and other advanced features.

\textcolor{red}{This text is red} and \textcolor{blue}{this text is blue} to test the \texttt{xcolor} package.

% ========================================
\section{Mathematical Expressions}
% ========================================

\subsection{Basic Math}

Here are some inline mathematical expressions: $E = mc^2$, $\pi \approx 3.14159$, and $\sum_{i=1}^{n} i = \frac{n(n+1)}{2}$.

\subsection{Display Math}

\begin{equation}
\int_{-\infty}^{\infty} e^{-x^2} dx = \sqrt{\pi}
\end{equation}

\begin{align}
f(x) &= ax^2 + bx + c \\
f'(x) &= 2ax + b \\
f''(x) &= 2a
\end{align}

\subsection{Advanced Math}

\begin{theorem}[Pythagorean Theorem]
In a right triangle with legs of length $a$ and $b$, and hypotenuse of length $c$:
\begin{equation}
a^2 + b^2 = c^2
\end{equation}
\end{theorem}

\begin{proof}
This is a well-known result. \qed
\end{proof}

Matrix example using \texttt{amsmath}:
\begin{equation}
\mathbf{A} = \begin{pmatrix}
a_{11} & a_{12} & \cdots & a_{1n} \\
a_{21} & a_{22} & \cdots & a_{2n} \\
\vdots & \vdots & \ddots & \vdots \\
a_{m1} & a_{m2} & \cdots & a_{mn}
\end{pmatrix}
\end{equation}

% ========================================
\section{Graphics and Figures}
% ========================================

\subsection{TikZ Graphics}

\begin{figure}[H]
\centering
\begin{tikzpicture}[node distance=2cm]
    \node (start) [draw, rectangle, fill=blue!20] {Start};
    \node (process) [draw, rectangle, fill=green!20, below of=start] {Process};
    \node (end) [draw, rectangle, fill=red!20, below of=process] {End};
    
    \draw [arrow, ->] (start) -- (process);
    \draw [arrow, ->] (process) -- (end);
\end{tikzpicture}
\caption{Simple TikZ flowchart}
\label{fig:tikz-example}
\end{figure}

% ========================================
\section{Tables}
% ========================================

\subsection{Basic Table}

\begin{table}[H]
\centering
\begin{tabular}{lcc}
\toprule
\textbf{Package} & \textbf{Purpose} & \textbf{Status} \\
\midrule
amsmath & Mathematical typesetting & \textcolor{green}{Working} \\
tikz & Graphics and diagrams & \textcolor{green}{Working} \\
hyperref & Hyperlinks & \textcolor{green}{Working} \\
\bottomrule
\end{tabular}
\caption{Package test results}
\label{tab:packages}
\end{table}

\subsection{Advanced Table}

\begin{table}[H]
\centering
\begin{tabular}{|l|c|r|p{3cm}|}
\hline
\multirow{2}{*}{\textbf{Feature}} & \multicolumn{2}{c|}{\textbf{Values}} & \multirow{2}{*}{\textbf{Notes}} \\
\cline{2-3}
& \textbf{Min} & \textbf{Max} & \\
\hline
Font Size & 10pt & 12pt & Standard sizes \\
Margins & 0.5in & 1.5in & Adjustable \\
\hline
\end{tabular}
\caption{Advanced table with multirow and multicolumn}
\end{table}

% ========================================
\section{Code Listings}
% ========================================

Here's a Python code example using the \texttt{listings} package:

\begin{lstlisting}[language=Python, caption=Python Example]
def fibonacci(n):
    """Generate Fibonacci sequence up to n terms."""
    if n <= 0:
        return []
    elif n == 1:
        return [0]
    elif n == 2:
        return [0, 1]
    
    fib = [0, 1]
    for i in range(2, n):
        fib.append(fib[i-1] + fib[i-2])
    
    return fib

# Test the function
print(fibonacci(10))
\end{lstlisting}

% ========================================
\section{Lists and Enumerations}
% ========================================

\subsection{Itemized List}

\begin{itemize}
    \item First item
    \item Second item with \textbf{bold text}
    \item Third item with \textit{italic text}
    \item Fourth item with \texttt{monospace text}
\end{itemize}

\subsection{Numbered List}

\begin{enumerate}
    \item First numbered item
    \item Second numbered item
        \begin{enumerate}
            \item Nested item A
            \item Nested item B
        \end{enumerate}
    \item Third numbered item
\end{enumerate}

\subsection{Custom Enumeration}

\begin{enumerate}[label=(\alph*)]
    \item Item with letter label
    \item Another item with letter label
    \item Yet another item
\end{enumerate}

% ========================================
\section{Hyperlinks and References}
% ========================================

This section tests the \texttt{hyperref} package:

\begin{itemize}
    \item Reference to Table~\ref{tab:packages}
    \item Reference to Figure~\ref{fig:tikz-example}
    \item External link: \url{https://www.latex-project.org/}
    \item Email link: \href{mailto:test@example.com}{test@example.com}
\end{itemize}

% ========================================
\section{Lorem Ipsum Text}
% ========================================

Testing the \texttt{lipsum} package for placeholder text:

\lipsum[1-2]

% ========================================
\section{Font Testing}
% ========================================

\subsection{Font Styles}

\begin{itemize}
    \item \textbf{Bold text}
    \item \textit{Italic text}
    \item \texttt{Typewriter text}
    \item \textsc{Small Caps Text}
    \item \underline{Underlined text}
    \item \textsl{Slanted text}
\end{itemize}

\subsection{Font Sizes}

{\tiny Tiny text} \\
{\scriptsize Script size text} \\
{\footnotesize Footnote size text} \\
{\small Small text} \\
{\normalsize Normal size text} \\
{\large Large text} \\
{\Large Larger text} \\
{\LARGE Even larger text} \\
{\huge Huge text} \\
{\Huge Hugest text}

% ========================================
\section{Special Characters and Symbols}
% ========================================

Testing various symbols and special characters:

\begin{itemize}
    \item Greek letters: $\alpha, \beta, \gamma, \delta, \epsilon, \pi, \sigma, \omega$
    \item Mathematical symbols: $\infty, \partial, \nabla, \sum, \prod, \int$
    \item Special characters: \&, \%, \$, \#, \_, \{, \}
    \item Accented characters: caf\'e, na\"ive, r\'esum\'e
\end{itemize}

% ========================================
\section{Conclusion}
% ========================================

This document successfully demonstrates the functionality of numerous \LaTeX\ packages commonly used with TeXShop. If you can compile this document without errors, your TeXShop installation is working correctly with these packages.

\subsection{Package Summary}

The following packages were tested in this document:

\begin{multicols}{2}
\begin{itemize}
    \item inputenc
    \item fontenc
    \item babel
    \item amsmath
    \item amssymb
    \item amsthm
    \item mathtools
    \item graphicx
    \item float
    \item subcaption
    \item tikz
    \item booktabs
    \item array
    \item longtable
    \item multirow
    \item xcolor
    \item fancyhdr
    \item geometry
    \item hyperref
    \item url
    \item cite
    \item listings
    \item verbatim
    \item enumerate
    \item enumitem
    \item lipsum
\end{itemize}
\end{multicols}

\end{document}
018